\documentclass{article}


\usepackage{bytefield}
\usepackage{tikz}
\usetikzlibrary{positioning}


\setlength{\parindent}{0pt}
\setlength{\parskip}{1em}


\begin{document}

\title{Single Cycle Computer}
\author{Jonathan Uhler}
\date{\today}
\maketitle

\textbf{Abstract}
This document describes the implementation of a basic single-cycle computer based on the instruction
set architecture defined in ``Instruction Set Architecture''.
\pagebreak


\section{Boot}
This section describes the initial state of the computer when power is connected up to the point
when the reset signal is released, allowing the program counter to increment on clock ticks.

\subsection{Hardware Boot}
When power is connected to the computer, no guarantee is made about the state of any memory,
including general purpose registers. However, the following must be true about the state of the
control and status registers:

\begin{itemize}
\item The \verb|reset| register must by set to \verb|0x0001|
\item The \verb|pc| register must be set to \verb|0x0100|
\end{itemize}

This will stop the clock from influencing the fetch-decode-exeute loop and will prepare the computer
to execute code at the reset vector.

\subsection{Software Boot}
The computer begins executing code at the absolute address in the program counter register when
\verb|reset| is ZERO. Reset must be released by an external actor (that is, not an instruction,
since no code can run while \verb|reset| is enabled).

The first instruction fetch, decode, and execute that occurs is at the reset vector (address
\verb|0x0100|). It is the responsibility of the code to initialize all general purpose registers and
memory as needed. This includes the stack pointer and return address.

A common control flow might look like:

\begin{itemize}
\item Code execution begins at the reset vector
\item The reset vector sets the stack pointer to the base of the stack (e.g. \verb|0xFFFE|)
\item The reset vector jumps to a \verb|_start| routine in the user code region (outside the
  vector table).
\item The \verb|_start| code performs any additional setup to run the main code
\item The main code runs and eventually returns to \verb|_start|, which executes \verb|halt|
\end{itemize}

\end{document}
